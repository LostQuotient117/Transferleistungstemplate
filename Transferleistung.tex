\documentclass[12pt,bibtotoc]{article}
\usepackage[table]{xcolor}
\usepackage[ngerman]{babel} % Für Silbentrennung
\usepackage[utf8]{inputenc} % Um Umlaute und Sonderzeichen nutzen zu können
\usepackage[a4paper,
lmargin={2cm},
rmargin={2cm},
tmargin={2cm},
bmargin={3cm}]{geometry} % Für das Layout
\usepackage{setspace} % Für den Zeilenabstand
\setstretch{1.25}
\usepackage{graphicx} % Um Bilder einzufügen
\usepackage{wrapfig} % Um Bilder im Fließtext einzubinden
\usepackage{csquotes} % Um Text in Anführungsstriche zu setzen
\usepackage{mathptmx} % Ähnlich zu Times New Roman
\usepackage{fancyhdr} % Um den Header einzufügen
\usepackage{pdfpages} % Um die Titelseite einbinden
\usepackage{hyperref} % Um Verlinkungen einfügen
\usepackage{acronym} % Um Akronyme zu definieren
\usepackage{textcomp} % Um spezielle Zeichen verwenden zu können (z.B. °, µ)
\usepackage{subcaption} % Um mehrere Bilder zusammen einzustellen und Links zu Abbildungen zu korrigieren
\usepackage{float}
\floatplacement{figure}{H}


\usepackage[backend=bibtex, citestyle=ieee]{biblatex} % Um Literatur einzufügen
\bibliography{Quellenverzeichnis}

\pagestyle{fancy}
\fancyhf{}
\setlength\headheight{59.76491pt}
\renewcommand{\headrulewidth}{0pt} % Um schwarze Linie zu entfernen
\chead{\includegraphics[width=\textwidth]{Resources/header.png}} % Grafik einfügen
\cfoot{\thepage}

\newcounter{romanBeginningEnd} % Um Römische Seitenzahl zu speichern

% Sektionen refernezen anpassen
\addto\extrasngerman{%
	\renewcommand{\sectionautorefname}{siehe}%
	\let\subsectionautorefname\sectionautorefname%
	\let\subsubsectionautorefname\sectionautorefname%
}



\begin{document}
    \pagenumbering{gobble}
    \newpage




\definecolor{color_30879}{rgb}{0,0.12549,0.376471}
\vspace{20mm}
\noindent
{\fontsize{15.96}{1}\usefont{T1}{cmr}{m}{n}\selectfont\color{color_30879}Transferleistung Theorie/Praxis  }\\ 
{\fontsize{15.96}{1}\usefont{T1}{cmr}{m}{n}\selectfont\color{color_30879}Nr. 2} 

\vspace{15mm}



\begin{center}
\begin{tabular}{ |>{\columncolor{color_30879}}p{1.6in} | p{4.4in}| } 
 \hline
 \textcolor{white}{Martrikelnummer:} & 12657 \\[0.2in]
 \hline
 \textcolor{white}{Freigegebenes Thema:} & Wie kann Knk das optimale Hosting-Modell auswählen, das den spezifischen Anforderungen und Zielen des Kunden entspricht, um Kosten zu optimieren, Flexibilität sicherzustellen und gleichzeitig Sicherheits- und Compliance-Anforderungen gerecht zu werden? \\ [1in]
 \hline
 \textcolor{white}{Studiengang, Zenturie:} & Wirtschaftsinformatik, I22c \\ [0.2in]
 \hline
\end{tabular}
\end{center}

	%\includepdf[pages=-]{Resources/cover}
	
	\setcounter{page}{1} % Römische Seitenzahlen
	\pagenumbering{Roman}
	
	\tableofcontents %Inhaltsverzeichnis
	\newpage
	
	\setcounter{secnumdepth}{0} % Keine Numerierung von Überschriften
	\clearpage
	\phantomsection
	\addcontentsline{toc}{section}{\listfigurename} % Abbildungsverzeichnis im Inhaltsverzeichnis anzeigen
	\listoffigures
	\clearpage
	\phantomsection
	%\addcontentsline{toc}{section}{\listtablename} % Tabellenverzeichnis im Inhaltsverzeichnis anzeigen
	%\listoftables
	\newpage
	\section{Abkürzungsverzeichnis}
	% Abkürzungen
	\begin{acronym}[LängsteAbkürzung] % Für Formatierung längste Abkürzung eintragen
	\acro{ac:Label}[knk]{knk Business Software AG}
	\acro{ac:Label}[SaaS]{Software as a Service}
	\end{acronym}
	\newpage %Ich werde ganz sicherlich keine Abkürzungen einbauen um meine Transferlistung mit weniger Seiten zu bestücken. DAFUQ
	
	\setcounter{secnumdepth}{3} % Überschriften wieder numerieren
	
	\setcounter{romanBeginningEnd}{\the\value{page}} % Speichern der römischen Seitenzahl
	\setcounter{page}{1} % Mit Arabischen Seitenzahlen wieder bei 1 anfangen
	\pagenumbering{arabic}
	
	% ###############################################################
	% Tipps und Tricks:
	%\section{Begriffserklärung}
	%Beispieltext
	%\subsection{Was ist der Fachkräftemangel}
	%\newpage
	
	
	%\begin{figure}[h!]
		%\includegraphics[keepaspectratio,width=\textwidth,height=\textheight]{statistaEntwicklungFachkraefteindex.png} \renewcommand{\figurename}{Abb.}
		%\caption{\small Entwicklung des Fachkr{\"a}fteindex in Deutschland vom 1. Quartal 2015 bis zum 4. Quartal 2022; Hays über Statista, 2023}
	%\end{figure}
	
	%\cite{knk-info.2023}
	
	%\newpage
	
	%	\begin{singlespace}
	%	Es muss insbesondere auf zwei zentrale Herausforderungen reagiert werden:
	%	\begin{itemize}
	%		\item stetig neue Mitarbeiter und Mitarbeiterinnen für sich zu gewinnen und
	%		\item bestehende Mitarbeiter möglichst langfristig an das eigene Unternehmen zu binden bzw. ihre Loyalit{\"a}t zu erhalten (wenn nicht angesichts kompetitiver Abwerbungsversuche sogar noch weiter zu erhöhen)
	%	\end{itemize}
	%\end{singlespace}
	
	%(\cite{Oelsnitz.2023}, Seite 4-5)
	
	%\begin{quote}
	%	,,Der aktuelle Wandel in Wirtschaft und Gesellschaft führt zu Umw{\"a}lzungen in einem bislang nicht gekannten Ausmaß. Strukturen, Prozesse, Aufgabenzuschnitte ver{\"a}ndern sich
	%	grundlegend – und dies in nahezu allen Branchen und Gesch{\"a}ftsbereichen.''\newline (\cite{Longmu.2021}, Seite 3)
	%\end{quote}

	%################################################################
	%Ab hier Inalt

	\section{Einleitung}
	%kleine Einleitung mit möglichen Methodenauflistung und einer groben Gliederungsbeschreibung?
	%möglicherweise eine Allgemeine Erklärung was Hosting-Modelle sind um den Leser bei den Anforderungsprofilen nicht zu überfordern?
	\subsection{Knk}
	\subsection{Hintergrund und Motivation}
	%möglicherweise erwähnen, dass es sich erstmal nur um BC dreht
	\subsection{Zielsetzung der Arbeit}
	%\subsection{Forschungsfrage} bietet aber vlt nicht genug


	\section{Anforderungsanalyse}
	\subsection{Erfassung der spezifischen Anforderungen von Knk und Kunden}
	%vlt auch subsubsection
	\subsection{Technische und gesellschaftliche Anforderungen}

	\section{Hosting-Modelle im Überblick}
		\subsection{On-Premise}
		\subsection{Cloud}
		\subsection{SaaS}
		
	\section{Detaillierter Vergleich der Hosting-Modelle}
		\subsection{Vor- und Nachteile jedes Modells}
		\subsection{Aktuelle Trends und Entwicklungen}

	\section{Entwicklung von Entscheidungskriterien}
		\subsection{Gewichtung der Kriterien nach Relevanz}
		\subsection{Unterstützung für Unternehmen und Kunden bei der Auswahl}

	\section{Best Practices und Fallbeispiele}
		\subsection{Praktische Beispiele aus der Industrie}
		\subsection{Empfehlungen für die Entscheidungsfindung}

	\section{Fazit}
		\subsection{Zusammenfassung der Ergebnisse}
		\subsection{Ausblick auf zukünftige Entwicklungen}
		%vlt dafür mal bei MS vorbei schauen. Die haben da bestimmt eine Planung




	
	% ###############################################################
	
	\newpage
	\pagenumbering{Roman}
	\setcounter{page}{\theromanBeginningEnd} % Römische Seitenzahlen fortsetzen
	\setcounter{secnumdepth}{0} % Nummerierung der Überschriften entfernt
	\section{Quellenverzeichnis}
	%\includepdf[pages=-]{Resources/cover}
	\setcounter{secnumdepth}{3} % Überschriften wieder numerieren (Für den Anhang)
	\printbibliography[heading=none]
	\newpage
	\appendix
	\clearpage
	\addcontentsline{toc}{part}{Anhang}
	%\includepdf[pages=-,
	%picturecommand*={\label{MEINPDFDOKUMENT}}]{Experteninterview mit Carsten Scheid ausgefüllt.pdf}

\end{document}