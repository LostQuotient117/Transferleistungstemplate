\documentclass[12pt,bibtotoc]{article}
\usepackage[table]{xcolor}
\usepackage[ngerman]{babel} % Für Silbentrennung
\usepackage[utf8]{inputenc} % Um Umlaute und Sonderzeichen nutzen zu können
\usepackage[a4paper,
lmargin={2cm},
rmargin={2cm},
tmargin={2cm},
bmargin={3cm}]{geometry} % Für das Layout
\usepackage{setspace} % Für den Zeilenabstand
\setstretch{1.25}
\usepackage{graphicx} % Um Bilder einzufügen
\usepackage{wrapfig} % Um Bilder im Fließtext einzubinden
\usepackage{csquotes} % Um Text in Anführungsstriche zu setzen
\usepackage{mathptmx} % Ähnlich zu Times New Roman
\usepackage{fancyhdr} % Um den Header einzufügen
\usepackage{pdfpages} % Um die Titelseite einbinden
\usepackage{hyperref} % Um Verlinkungen einfügen
\usepackage{acronym} % Um Akronyme zu definieren
\usepackage{textcomp} % Um spezielle Zeichen verwenden zu können (z.B. °, µ)
\usepackage{subcaption} % Um mehrere Bilder zusammen einzustellen und Links zu Abbildungen zu korrigieren
\usepackage{float}
\floatplacement{figure}{H}


\usepackage[backend=bibtex, citestyle=ieee]{biblatex} % Um Literatur einzufügen
\bibliography{Quellenverzeichnis}

\pagestyle{fancy}
\fancyhf{}
\setlength\headheight{59.76491pt}
\renewcommand{\headrulewidth}{0pt} % Um schwarze Linie zu entfernen
\chead{\includegraphics[width=\textwidth]{Resources/header.png}} % Grafik einfügen
\cfoot{\thepage}

\newcounter{romanBeginningEnd} % Um Römische Seitenzahl zu speichern

% Sektionen refernezen anpassen
\addto\extrasngerman{%
	\renewcommand{\sectionautorefname}{siehe}%
	\let\subsectionautorefname\sectionautorefname%
	\let\subsubsectionautorefname\sectionautorefname%
}



\begin{document}
    \pagenumbering{gobble}
    \newpage




\definecolor{color_30879}{rgb}{0,0.12549,0.376471}
\vspace{20mm}
\noindent
{\fontsize{15.96}{1}\usefont{T1}{cmr}{m}{n}\selectfont\color{color_30879}Transferleistung Theorie/Praxis  }\\ 
{\fontsize{15.96}{1}\usefont{T1}{cmr}{m}{n}\selectfont\color{color_30879}Nr. 2} 

\vspace{15mm}



\begin{center}
\begin{tabular}{ |>{\columncolor{color_30879}}p{1.6in} | p{4.4in}| } 
 \hline
 \textcolor{white}{Martrikelnummer:} & 12657 \\[0.2in]
 \hline
 \textcolor{white}{Freigegebenes Thema:} & Wie kann Knk das optimale Hosting-Modell auswählen, das den spezifischen Anforderungen und Zielen des Kunden entspricht, um Kosten zu optimieren, Flexibilität sicherzustellen und gleichzeitig Sicherheits- und Compliance-Anforderungen gerecht zu werden? \\ [1in]
 \hline
 \textcolor{white}{Studiengang, Zenturie:} & Wirtschaftsinformatik, I22c \\ [0.2in]
 \hline
\end{tabular}
\end{center}

	%\includepdf[pages=-]{Resources/cover}
	
	\setcounter{page}{1} % Römische Seitenzahlen
	\pagenumbering{Roman}
	
	\tableofcontents %Inhaltsverzeichnis
	\newpage
	
	\setcounter{secnumdepth}{0} % Keine Numerierung von Überschriften
	\clearpage
	\phantomsection
	\addcontentsline{toc}{section}{\listfigurename} % Abbildungsverzeichnis im Inhaltsverzeichnis anzeigen
	\listoffigures
	\clearpage
	\phantomsection
	%\addcontentsline{toc}{section}{\listtablename} % Tabellenverzeichnis im Inhaltsverzeichnis anzeigen
	%\listoftables
	\newpage
	\section{Abkürzungsverzeichnis}
	% Abkürzungen
	\begin{acronym}[LängsteAbkürzung] % Für Formatierung längste Abkürzung eintragen
	\acro{ac:Label}[knk]{knk Business Software AG}
	\acro{ac:Label}[SaaS]{Software as a Service}
	\acro{ac:Label}[CRM]{Customer Relationship Management}
	\end{acronym}
	\newpage %Ich werde ganz sicherlich keine Abkürzungen einbauen um meine Transferlistung mit weniger Seiten zu bestücken. DAFUQ
	
	\setcounter{secnumdepth}{3} % Überschriften wieder numerieren
	
	\setcounter{romanBeginningEnd}{\the\value{page}} % Speichern der römischen Seitenzahl
	\setcounter{page}{1} % Mit Arabischen Seitenzahlen wieder bei 1 anfangen
	\pagenumbering{arabic}
	
	% ###############################################################
	% Tipps und Tricks:
	%\section{Begriffserklärung}
	%Beispieltext
	%\subsection{Was ist der Fachkräftemangel}
	%\newpage
	
	
	%\begin{figure}[h!]
		%\includegraphics[keepaspectratio,width=\textwidth,height=\textheight]{statistaEntwicklungFachkraefteindex.png} \renewcommand{\figurename}{Abb.}
		%\caption{\small Entwicklung des Fachkr{\"a}fteindex in Deutschland vom 1. Quartal 2015 bis zum 4. Quartal 2022; Hays über Statista, 2023}
	%\end{figure}
	
	%\cite{knk-info.2023}
	
	%\newpage
	
	%	\begin{singlespace}
	%	Es muss insbesondere auf zwei zentrale Herausforderungen reagiert werden:
	%	\begin{itemize}
	%		\item stetig neue Mitarbeiter und Mitarbeiterinnen für sich zu gewinnen und
	%		\item bestehende Mitarbeiter möglichst langfristig an das eigene Unternehmen zu binden bzw. ihre Loyalit{\"a}t zu erhalten (wenn nicht angesichts kompetitiver Abwerbungsversuche sogar noch weiter zu erhöhen)
	%	\end{itemize}
	%\end{singlespace}
	
	%(\cite{Oelsnitz.2023}, Seite 4-5)
	
	%\begin{quote}
	%	,,Der aktuelle Wandel in Wirtschaft und Gesellschaft führt zu Umw{\"a}lzungen in einem bislang nicht gekannten Ausmaß. Strukturen, Prozesse, Aufgabenzuschnitte ver{\"a}ndern sich
	%	grundlegend – und dies in nahezu allen Branchen und Gesch{\"a}ftsbereichen.''\newline (\cite{Longmu.2021}, Seite 3)
	%\end{quote}

	%################################################################
	%Ab hier Inalt

	\section{Einleitung}
	%kleine Einleitung mit möglichen Methodenauflistung und einer groben Gliederungsbeschreibung?
	%möglicherweise eine Allgemeine Erklärung was Hosting-Modelle sind um den Leser bei den Anforderungsprofilen nicht zu überfordern?
	\subsection{Knk}
	Die Knk-Gruppe ist ein norddeutsches, international agierendes Unternehmen mit Hauptsitz in Kiel. Sie setzt sich aus den Unternehmen knk Business Software AG, der Business Unit muellerPrange, der knk Customer Engagement GmbH, der knk Cloud Services GmbH und Bradbury Phillips International zusammen. Hinzu kommen die knk Software LP (USA), knk Software Ltd. (UK) und knk France SAS (Frankreich)\cite{knk-info.2023}.
	\newline \newline
	Mit ihren Lösungen und Services innerhalb der Gruppe unterstützen sie Verlage und Medienunternehmen dabei, die Chancen der Digitalisierung und aktuelle Entwicklungen der Branche zu nutzen, Arbeitsabläufe zu optimieren und neue Zielgruppen zu erreichen. Im Fokus stehen hierbei neue Content-basierte Geschäftsmodelle, Business Intelligence und Künstliche Intelligenz für Verlage, CRM, Social Media Marketing sowie Marketing Automation\cite{knk-info.2023}.
	Das Produkt der knk Gruppe besteht aus dem Grundmodell von Business Central aus dem Hause Microsoft und wird in der ,,AL-Programming Language'' entwickelt und erweitert.
	\subsection{Hintergrund, Motivation und Zielsetzung der Transferleistung}
	Knk bietet schon lange Großprojekte für Medien- und Verlagshäuser in Form von KnkVerlag an. 
	Um nun auch kleinere Verlagshäuser anzusprechen, wurde knkMedia SaaS angekündigt und zur Verfügung gestellt \cite{knk-kuendigtSaaS-an.2023}.
	\newline
	knkMedia Saas soll schnelle aktualisierung innerhalb von Stunden ermöglichen und über die Microsoft-Cloud verfügbar sein.
	Die Vermarktung von knkMedia Saas als fertiges Produktpacket soll nun auch kleineren Verlagen mit 10 bis 20 Benutzern das ERP-System von Microsoft und die dazu passenden Knk-Erweiterungen attraktiver gestalten \cite{knk-kuendigtSaaS-an.2023}.
	\newline
	Die Motivation dieser Arbeit besteht darin, einen genaueren Überblick auf die verschiedenen Hostingmodelle zu bekommen und einen möglichst genauen Leitfaden zur Entscheidungsfindung aufzubauen.
	Der daraus entstehende Leeitfaden soll ermöglichen besser über Vor- und Nachteile informiert zu sein und bessere Entscheidungen aufgrund dieser zu treffen. 
	\newline
	Diese Transferleistung wird zwar auch allgemeine Fakten über Hosting-Modelle offenlegen, jedoch sich stark an Microsoft-Dienste orientieren um das von Microsoft entwickelte und von Knk erweiterte Business Central ERP-System 
	thematisch in den Vordergrund zu rücken. Zudem vertreibt Knk ihr Produkt nur in Verbindung mit Business Central.
	%möglicherweise erwähnen, dass es sich erstmal nur um BC dreht (check)
	%\subsection{Forschungsfrage} bietet aber vlt nicht genug


	\section{Anforderungsanalyse}
	\subsection{Erfassung der spezifischen Anforderungen von Knk und Kunden}
	%vlt auch subsubsection
	\subsection{Technische und gesellschaftliche Anforderungen}

	\section{Hosting-Modelle im Überblick}
		\subsection{On-Premise}
		Bei einem On-Premise-Modell handelt es sich um eine lokal installierte Software, die auf jedem Computer eines Benutzers installiert werden muss, um die gewünschten Funktionalitäten zur Verfügung zu stellen.
		Meist sind auch die Daten, die eine Unternehmung für diese On-Premise-Sofware gebraucht auf lokalen oder Firmenweiten Datenbanken gespeichert \cite{AlHayek.2020}. 
		\subsection{Cloud}
		%https://www.leanix.net/de/wiki/apm/saas-vs-cloud#What-is-cloud-based  Cloud ist überbegriff und SaaS ist eine unterkategorie. Das im SaaS-Kapitel hervorzeigen. 
		Cloud erntet immer mehr aufmerksamkeit und wird weitesgehend adaptiert. Als paradigmenwechsel in der IT betrachtet, wird es schon für vielfältige Anwedungsbereiche sowohl im geschäftlichen aber auch im privaten und behördlichen Bereich angewendet. \cite{Murugesan.2016}.
		\newline
		Die meisten privaten Nutzer, gebrauchen Cloud bereits um Fotos zu speichern oder für den eigenen Online-Kalendar sowie Online-Datenspeicherungen (z.B.: OneDrive)
		Klein- und Mittelgroße Unternehmen gebrauchen Cloud-Lösungen zum Beispiel für Cloud-basierte Anwedungen, Gehaltsabrechnungen, Kundenbeziehungsmanagement (CRM), Business Intelligence oder Datensammlung und Analyse.
		Großunternehmen nutzen Cloud-Dienste zum Beispiel für Geschäftsfunktionen wie Supply-Chain-Management, Datenspeicherung, Big-Data-Analysen, Geschäftsprozessmanagement, CRM oder Anwendungsentwicklung \cite{Murugesan.2016}.
		%nochmal sagen: Jo meist einfach Browser an und fertig der Lachs
		%\begin{singlespace} 
		\newline
			Es muss bei Cloud auf folgende Grundeigenschaften verwiesen werden \cite{Murugesan.2016}.
			\begin{itemize}
				\item Clouds stehen massive Ressourcen bei Bedarf zur Verfügung.
				\item Ressourcen können bedarfsgerecht skaliert werden.
				\item Sie bieten Ubiquität. Dies bedeutet, dass Daten unabhängig von Standort, Benutzer und Tageszeit zugänglich gemacht werden können.
				\item Es wird die gemeinsame Nutzung von Daten und unternehmensweite Datenanalyse sowie Zusammenarbeit gefördert.
				\item Clouds können sich bei Bedarf selbst rekonfigurieren um die Verfügbarkeit im Falle eines Ausfalls ihrer Datenverarbeitungsressourcen zu gewährleisten.
				\item Sie bieten eine vereinfachte Web-Browser-Oberfläche.
			\end{itemize}
		\newpage
		%\end{singlespace}
		%herausfinden wie genau wir Cloud einsetzen, wer es dann hostet und wie es funktioniert und inwiefern es sich von SaaS abgrenzt.
		%auf jedenfall erwähnen seit wann Microsoft auch Cloud für BC anbietet
		\subsubsection{SaaS}
		Software as a Service, ist eine bestimmte Art der Nutzung von Cloud die auch Software-Clouds genannt werden. 
		Beim SaaS-Modell wird eine Anwendung von einem Cloud-Anbieter gehostet und als Service für die Nutzer bereitgestellt, in erster Linie über das Internet oder ein spezielles Netz.
		Es entfällt dadurch die Notwendigkeit, die Anwedung lokal auf dem Computer des Nutzers zu installieren und auszuführen und die Nutzer müssen sich nicht mehr um die Wartund und Aktualisierung der Hardware und Software kümmern.
		Den Nutzern wird dabei die genutzten Dienstleistungen in Rechnung gestellt. Somit werden die Kosten für die Nutzung eines Dienstes zu einer kontinuierlichen Ausgabe und nicht zu einer großen Anfangsinvestition zum Zeitpunkt des Kaufs \cite{Murugesan.2016}.
		%sind unterkategorie von Cloud-Lösungen. 
		
	\section{Detaillierter Vergleich der Hosting-Modelle}
			Durch die stationäre Bereitstellung der Umgebung und einer erforderten Installation des Produktes auf jedem Gerät ist bei On-Premise eine gewisse Flexibilität, nicht gegeben.
			Änderungen an der Software sorgen meist für einen gewissen Mehraufwand auf Seiten des Kunden und des Entwicklers. Dieser Mehraufwand ist bei Cloud nicht gegeben durch ein einfach nutzbares Browser-Interface. 
			On-Premise beinhaltet daher Kostenkomponenten, die eine Eigenverantwortung für Faktoren wie Einrichtung des Servers vor Ort, Serversoftware, Arbeitsaufwand für die Systemadministatoren und andere Infrasrukturkosten beinhaltet, welche bei einem Cloud-Service in der Regel nicht anfallen \cite{Fisher.2018}.
			\newline
			Sicherheit und Datenschutz ist ein weiterer Aspekt. Durch die lokale Speicherung der Daten bei On-Premise kann eine gewisse Macht über die eigenen Daten ausgeübt werden. 
			Bei Cloud-Lösungen wird diese Macht jedoch an den Host der Cloud-Server-Host abgegeben der die Server zur Datenspeicherung besitzt. Dies kann im subjektiven Fall ein Problem darstellen. 
			\newline
			Die Skalierung der Ressourcen hebt Cloud maßgeblich von On-Prem ab. On-Prem bietet mit seinen lokalen Servers eine meist feststehende beziehungsweise schwer zu erweiterbare Ressourcenstruktur. Braucht eine Unternehmung mit einem On-Prem-System eine 
			Ressourcenerweiterung, muss meist teure Hardware nachgekauft werden, was eine hohe einmalige Investition bedeutet.
			Bei Cloud wiederum, muss der Anbieter mehr Ressourcen frei geben, was zwar wieder eine höhere Bindung an den Cloud-Host bedeutet, jedoch auch eine vermeidung von Investitionen und je nach Anbieter lediglich eine Erhöhung der Gebühren zur Folge hat \cite{Murugesan.2016}.
			
			\subsection{Aktuelle Trends und Entwicklungen}

	\section{Entwicklung von Entscheidungskriterien}
	Zunächst werden Kriterien aufgestellt und erläutert:
	\begin{itemize}
		\item User-Anzahl
		\begin{itemize}
			\item Die Anzahl der User innerhalb eines Unternehmens...
		\end{itemize}
	\end{itemize}
		\subsection{Gewichtung der Kriterien nach Relevanz}
		\subsection{Unterstützung für Unternehmen und Kunden bei der Auswahl}

	\section{Best Practices und Fallbeispiele}
		\subsection{Praktische Beispiele aus der Industrie}
		\subsection{Empfehlungen für die Entscheidungsfindung}

	\section{Fazit}
		\subsection{Zusammenfassung der Ergebnisse}
		\subsection{Ausblick auf zukünftige Entwicklungen}
		%vlt dafür mal bei MS vorbei schauen. Die haben da bestimmt eine Planung




	
	% ###############################################################
	
	\newpage
	\pagenumbering{Roman}
	\setcounter{page}{\theromanBeginningEnd} % Römische Seitenzahlen fortsetzen
	\setcounter{secnumdepth}{0} % Nummerierung der Überschriften entfernt
	\section{Quellenverzeichnis}
	%\includepdf[pages=-]{Resources/cover}
	\setcounter{secnumdepth}{3} % Überschriften wieder numerieren (Für den Anhang)
	\printbibliography[heading=none]
	\newpage
	\appendix
	\clearpage
	\addcontentsline{toc}{part}{Anhang}
	%\includepdf[pages=-,
	%picturecommand*={\label{MEINPDFDOKUMENT}}]{Experteninterview mit Carsten Scheid ausgefüllt.pdf}

\end{document}